\section{Design}
I detta avsnitt avhandlas ramverket och dess beståndsdelar samt ett litet
tillägg som gjordes för att möjliggöra förandet av statistik över vilka sidor
som besöks.

\subsubsection*{En löst specificerad design}
\marginpar{Bättre rubrik efterfrågas...}
I början av arbetet med projektet hade vi inte en klar bild av hur ramverkets
arkitektur skulle se ut, men för att påskynda arbetet implementerades den
arkitektur som beskrivs i resten av dokumentet. Det som först implementerades
var en simpel version av det som beskrivs i resten av det här avsnittet. Under
vårt arbete utökade vi ramverket med den funktionalitet vi kom att behöva och
nådde till slut den färdiga produkten.

\subsection{Moduler}
\marginpar{\footnotesize Är detta svammel? :P}
Moduler spelar en central roll i ramverket eftersom de är byggstenarna i en sida
som byggs med Tekweb. En modul är i grunden en klass vars instans(er) har
för syfte att sammanställa och göra information användbar för en besökare.
Informationen kan läsas in på flera sätt, Tekweb erbjuder ett simpelt gränssnitt
för att läsa ut information från andra hemsidor som en lång sträng och om
behovet finns så kan man även implementera hjälpklasser för att göra mer
avancerad datainsamling (med lämpliga ändringar av informationen), för att
hålla modulklassens komplexitet på en rimlig nivå.

Då moduler måste ärva av den abstrakta klassen ContentModule\footnote{
$<$Referens till källkods-filen som innehåller ContentModule?$>$} så kommer de
i och med arvet att innehålla de grundläggande funktioner som behövs för att
modulen ska kunna interagera med ramverket. Det enda en
modulklass ansvarar för är att vissa medlemsvariabler tilldelas lämpliga
värden, vilket låter oss skriva enklare kod som fortfarande går att överblicka
när modulen är färdig.

Modulernas innehåll renderas i slutändan på ett av de sätt (om det är
implementerat i modulen) som finns i figur~\ref{list:modes}, beroende på vilket
tillstånd modulen befinner sig i.

\begin{figure}[h]
\begin{itemize}
  \item[{\bf Standardläget (default)}] används typiskt för att visa en
    sammanställning av all information som finns tillgänglig för användaren.
  \item[{\bf Växelläget (toggler)}] används när man vill ha ett klickbart objekt
    på hemsidan som vecklar ut en ruta med det önskade innehållet när den
    aktiveras.
  \item[{\bf Frestelseläget (teaser)}] används då man vill visa en länk som ska
    fresta användaren till att surfa vidare, t.ex. då man visar upp en rubrik
    till en nyhet.
\end{itemize}
\caption{En lista av möjliga tillstånd för moduler\label{list:modes}}
\end{figure}

För att minimera trafikvolymen mellan telefonen och servern valde vi att
implementera ett gränssnitt med hjälp av AJAX och i vissa fall en variant av
AJAX som inte kretsar kring XML.\footnote{Vi borde kanske nämna AJAX någonstans
så att det inte är ryckt ur luften} Detta är en av funktionerna vi såg ett behov
av under arbetet med projektet. Mer om detta under rubrik~\ref{sec:ui}.

\subsubsection*{Hjälpfiler}
Utöver den PHP-fil som modulenklassen är implementerad i så används i vissa fall
även någon/några av följande hjälpfiler;

\begin{itemize}
  \item PHP-fil(er) med hjälpklasser.
  \item JavaScript-fil(er) som innehåller funktionalitet som behövs hos
    klienten.
  \item XSL-mall(ar) som bestämmer hur modulens generererade XML-kod ska
    transformeras till HTML.
  \item CSS-fil(er) som används för att definiera hur den genererade HTML-koden
    ska presenteras för användaren.
\end{itemize}

\subsection{XML-konfigurationen}
Ramverket konfigureras med en XML-fil som är uppbyggd enligt
figur~\ref{fig:xml-config}.

\begin{figure}[h]
\begin{verbatim}
                         root
                 _________|_______...__
                 |     |     |        |
               title  item  item     item
                       |     |        |
                       .     .        .
                       .     .        .
                       .     .        .
\end{verbatim}
\caption{Illustration av konfigurationsfilen}
\label{fig:xml-config}
\end{figure}

Rot-noden ({\it root}) innehåller alltså en titel-nod ({\it title}) och ett
godtyckligt antal modul-noder ({\it item}). I {\it title} finns en sträng som
senare används för att sätta sidans titel.

{\it item} används för att beskriva attribut hos moduler som ska finnas på
sidan, de måste åtminstone innehålla {\it module} och {\it mode}. De måste även
innehålla {\it settings/name} om man vill att ramverket ska kunna utföra
operationer som kräver att modulen kan pekas ut.

\subsection{XSLT-mallar}


...används för att transformera data i mellan mellanrepresentationer och för att
generera den text som till slut levereras till användarens webbläsare.

\subsection{Ramverket}
\marginpar{\footnotesize Ska vi nämna inbyggda funktioner här?}

Den centrala delen av ramverket, som sköter genereringen av innehållet, ligger
i $htdocs/index.php$ och är i grunden uppbyggd av tre steg:

\begin{itemize}
  \item[\bf Förberedelsen]
    I detta steg förbereds PHP-miljön för att kunna exekvera modulerna genom
    att:
    \begin{inparaenum}[\itshape a\upshape)]
      \item länka in nödvändig funktionalitet som finns definierad i
        andra filer,
      \item eventuellt trimma den laddade konfigurationen,
      \item ladda konfigurationen och
      \item ladda huvudsidans mall,
    \end{inparaenum}
    för att kunna generera huvudsidan.

    I vissa fall kan förfrågan som för tillfället besvaras vara endera en
    AJAX-fråga eller en förfrågan om att få en viss undersida. I dessa fall
    kommer det att göras en kontroll för att avgöra huruvida det är en giltig
    förfrågan. Om frågan visar sig vara giltig så kommer den laddade
    konfigurationen att anpassas därefter.
  \item[\bf Huvud-loopen]
    Huvud-loopen.

  \item[\bf Utskriften]
    I detta steg skrivs det genererade innehållet helt enkelt ut.
\end{itemize}

\subsection{Användargränssnitt}
\label{sec:ui}
Här skriver vi mer om hur vi tänkte med UI:et och varför det är ssssmmmmexxayyy!

\subsection{Utbyggnader}
\label{design:stats}
Här skriver Magnus några rader om hur han haxX0rade in statistik-fixen.

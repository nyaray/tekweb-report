\section{Design}
I detta avsnitt avhandlas ramverkets moduler, konfiguration, mallar, ramverket
självt samt ett litet tillägg om hur en mindre utbyggnad gjordes för att
möjliggöra förandet av statistik över vilka sidor som besöks.

\subsection{Moduler}
Moduler spelar en central roll i ramverket 

Här skriver vi om modulerna, hur de förhåller sig till varandra, systemet samt
lite om arv. Det är viktigt att vi poängterar hur centrala moduler är men kanske
ännu viktigare, varför de är så centrala som de är.

\subsection{XML-konfigurationen}
Ramverket konfigureras med en XML-fil som har följande struktur:

\begin{verbatim}
root
|_ title
|_ item_0
   |_ ...
|_ item_1
   |_ ...
.
.
.
|_ item_n
   |_ ...
\end{verbatim}

Rot-noden innehåller alltså en {\it title}-nod och ett godtyckligt antal
{\it item}-noder. {\it title}-noden innehåller en sträng som senare används för
att sätta sidans titel. {\it item}-noder används för att beskriva instanser av
moduler som ska finnas på sidan.

\subsection{XSLT-mallar}
...används för att transformera data i mellan mellanrepresentationer och för att
generera den text som till slut levereras till användarens webbläsare.

\subsection{Ramverket}
Den centrala delen av ramverket, som sköter genereringen av innehållet, ligger
i $htdocs/index.php$ och är i grunden uppbyggd av tre steg:

\begin{itemize}
  \item[\bf Förberedelsen]
    I detta steg förbereds PHP-miljön för att kunna exekvera modulerna genom
    att:
    \begin{inparaenum}[\itshape a\upshape)]
      \item länka in nödvändig funktionalitet som finns definierad i
        andra filer,
      \item eventuellt trimma den laddade konfigurationen,
      \item ladda konfigurationen och
      \item ladda huvudsidans mall,
    \end{inparaenum}
    för att kunna generera huvudsidan.

    I vissa fall kan förfrågan som för tillfället besvaras vara endera en
    AJAX-fråga eller en förfrågan om att få en viss undersida. I dessa fall
    kommer det att göras en kontroll för att avgöra huruvida det är en giltig
    förfrågan. Om frågan visar sig vara giltig så kommer den laddade
    konfigurationen att anpassas därefter.
  \item[\bf Huvud-loopen]
    Huvud-loopen.

  \item[\bf Utskriften]
    I detta steg skrivs det genererade innehållet helt enkelt ut.
\end{itemize}

\subsection{Utbyggnader}
Här skriver Magnus några rader om hur han haxX0rade in statistik-fixen.

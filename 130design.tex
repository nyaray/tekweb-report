\section{Design}
\marginpar{\footnotesize Borde vi även nämna när det är (o)lämpligt att använda
Tekweb?}

I detta avsnitt avhandlas ramverkets moduler, konfiguration, mallar, ramverket
självt samt ett litet tillägg om hur en mindre utbyggnad gjordes för att
möjliggöra förandet av statistik över vilka sidor som besöks.

Den ursprungliga designen utformades primärt med PHP-klasserna i åtanke,
men utvecklades under arbetets allteftersom PHP-klasserna bröts ned i bättre,
mer logiska enheter.

\subsection{Moduler}
\marginpar{\footnotesize Är detta svammel? :P}
Moduler spelar en central roll i ramverket eftersom de är byggstenarna i en sida
som byggs med Tekweb. En modul är i grunden en klass vars instans(er) har
för syfte att sammanställa och göra information användbar för en besökare.
Informationen kan läsas in på flera sätt, Tekweb erbjuder ett simpelt gränssnitt
för att läsa ut information från andra hemsidor som en lång sträng och om
behovet finns så kan man även implementera hjälpklasser för att göra mer
avancerad datainsamling (med lämpliga ändringar av informationen), för att
hålla modulklassens komplexitet på en rimlig nivå.

Eftersom moduler ärver av den abstrakta klassen {\bf ContentModule} så kommer de
även att innehålla de grundläggande funktioner som behövs. Det enda en
modulklass ansvarar för är att vissa medlemsvariabler får lämpliga värden, den
abstrakta klassen sköter överlämningen av informationen till den del av
ramverket som instansierat modulen. Detta låter oss skriva enklare, men
framförallt tydligare, kod som fortfarande går att överblicka när en modul är
färdig.

\subsubsection{Hjälpfiler}
Utöver den PHP-fil som modulenklassen är implementerad i så används i vissa fall
även en av följande hjälpfiler;

\begin{itemize}
  \item PHP-fil(er) med hjälpklasser.
  \item JavaScript-fil(er) som innehåller funktionalitet som behövs hos
    klienten.
  \item XSL-mall(ar) som bestämmer hur modulens generererade XML-kod ska
    transformeras till HTML.
  \item CSS-fil(er) som används för att definiera hur den genererade HTML-koden
    ska presenteras för användaren.
\end{itemize}

\subsection{XML-konfigurationen}
Ramverket konfigureras med en XML-fil som är uppbyggd enligt
figur~\ref{fig:xml-config}.

\begin{figure}[h]
\begin{verbatim}
root
|_ title
|_ item
   |_ ...
|_ item
   |_ ...
.
.
.
|_ item
   |_ ...
\end{verbatim}
\caption{Illustration av konfigurationsfilen}
\label{fig:xml-config}
\end{figure}

Rot-noden innehåller alltså en {\it title}-nod och ett godtyckligt antal
{\it item}-noder. {\it title}-noden innehåller en sträng som senare används för
att sätta sidans titel. {\it item}-noder används för att beskriva instanser av
moduler som ska finnas på sidan.

\subsection{XSLT-mallar}
...används för att transformera data i mellan mellanrepresentationer och för att
generera den text som till slut levereras till användarens webbläsare.

\subsection{Moduler}
Här skriver vi om modulerna, hur de förhåller sig till varandra, systemet samt
lite om arv.

Det blir konstigt att vi kan så mycket om moduler i ramverket om vi
beskriver ramverket innan vi beskriver moduler.

\subsection{Ramverket}
\marginpar{\footnotesize Ska vi nämna inbyggda funktioner här?}

Den centrala delen av ramverket, som sköter genereringen av innehållet, ligger
i $htdocs/index.php$ och är i grunden uppbyggd av tre steg:

\begin{itemize}
  \item[\bf Förberedelsen]
    I detta steg förbereds PHP-miljön för att kunna exekvera modulerna genom
    att:
    \begin{inparaenum}[\itshape a\upshape)]
      \item länka in nödvändig funktionalitet som finns definierad i
        andra filer,
      \item ladda konfigurationen och
      \item ladda huvudsidans mall,
    \end{inparaenum}
    för att kunna generera huvudsidan.

    I vissa fall kan förfrågan som för tillfället besvaras vara endera en
    AJAX-fråga eller en förfrågan om att få en viss undersida. I dessa fall
    kommer det att göras en kontroll för att avgöra huruvida det är en giltig
    förfrågan. Om frågan visar sig vara giltig så kommer den laddade
    konfigurationen att anpassas därefter.
  \item[\bf Huvud-loopen]
    Huvud-loopen.

  \item[\bf Utskriften]
    I detta steg skrivs det genererade innehållet helt enkelt ut.
\end{itemize}

\subsection{Utbyggnader}
Här skriver Magnus några rader om hur han haxX0rade in statistik-fixen.

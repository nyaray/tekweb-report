\subsection{Modulerna}
De moduler som i dagsläget finns implementerade är, utan någon särskild
ordning;

\begin{itemize}
\item[\bf{about}] används för att visa en ``om''-text, d.v.s. en kort text om
hemsidan;

\item[\bf{empsearch}] låter användaren söka efter personer på en LDAP-server;

\item[\bf{feed}] läser ut RSS-flöden (även andra sorter((????))) och
presenterar dem på ett trevligt sätt;

\item[\bf{multifeed}] kombinerar \emph{feed}-instanser för att aggregera
innehåll från olika källor till ett uniformt flöde;

\item[\bf{statictext}] presenterar statiskt innehåll för användaren;

\item[\bf{timeedit}] ligger som ett lager ovanpå det befintliga
TimeEdit-systemet och gör ett mobilvänligt användargränssnitt.

\item[\bf{uumap}] visar busshållsplatser, nationshus, m.m. Filip får skriva
här, han gjorde maaaassoooor!

\end{itemize}

% skapa filer enligt 1420modul.tex, använd input för att få in dem i
% dokumentet, skippa filändelsen. alla har 1420 som prefix så att vi kan ordna
% dem på något sätt sen.

% skapa en branch för den fil du tänker jobba på, lägg inte till den i någon
% annan gren än den du jobbar i, t.ex. about eller feed. om du behöver se hur
% dokumentet ser ut så tar du och gör en checkout av implementation, tar bort
% kommentaren för den modulen du vill titta på, mergar in i implementation och
% (obs!) PUSHAR ALDRIG NÅGONSIN (obs!) din nya version av implementation till
% github.

%\input{1420about}
%\input{1420empsearch}
%\input{1420feed}
%\input{1420multifeed}
%\input{1420statictext}
%\input{1420timeedit}
%\input{1420uumap}


\section{Ramverket}
Ramverket är skrivet i PHP och förlitar sig på följande
moduler/bibliotek;

\begin{itemize}
  \item[{\bf Output Buffering}] Utskrifter som vanligen hade gått till
    klientens webbläsare går istället till en buffert för mellanlagring.

  \item[{\bf Session}] Används för att lagra sessionsspecifik data.

  \item[{\bf DOM}] För att arbeta med konfigurationsfilen och intern data.

  \item[{\bf XSL}] För att transformera data som läses ut från andra tjänster
    på internet och i universitetets nät.
\end{itemize}

\subsection{Filträd}
Tekwebs filer lagras enligt följande struktur;

\begin{verbatim}
 /
|- conf/
| |- config.php
| |- root.xml
| |- root.xsl
|- css/
| |- layout.css
| |- module/
| | |- <modulspecifika css-filer>
| |- reset.css
| |- style.css
|- gfx/
| |- <instansspecifika bilder>
| |- icons/
| | |- <modulikoner>
| |- module/
|   |- <modulkataloger>
|- htdocs/
| |- index.php
|- include/
| |- <modulers hjälpbibliotek>
| |- contentmodule.php
| |- core.php
|- index.php
|- js/
| |- <modulers javascript-filer> 
| |- main.js
|- module/
| |- <modulkataloger>
|   |- <modul>.php
|   |- [<xsl-mall>.xsl]
|- tekweb.tmproj
\end{verbatim}
